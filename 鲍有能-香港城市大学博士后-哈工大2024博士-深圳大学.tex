% !TeX encoding = UTF-8
% !TeX program = xelatex
% !TeX spellcheck = en_US
% choose Xelatex compiler 选择Xelatex进行编译
\documentclass{resume}
\usepackage{zh_CN-Adobefonts_external} 
\usepackage{linespacing_fix}
\usepackage{cite}
\usepackage{comment}
\usepackage{enumitem}
\usepackage{hyperref}
\usepackage{fontspec}
\usepackage{fontawesome}
\usepackage{xcolor}

% \usepackage{emoji}
% \setmainfont{DejaVu Sans} % 确保使用支持emoji的字体
% you should read https://www.overleaf.com/learn/latex/Learn_LaTeX_in_30_minutes to learn basic latex

% read manual https://github.com/xiangrongjingujiu/latex-languageSelection
\usepackage[Chinese]{languageSelection}
% \usepackage[English]{languageSelection}

% read manual https://github.com/xiangrongjingujiu/latex-note-plus
\usepackage[color=blue]{notePlus} 


\begin{document}
\pagenumbering{gobble}

% "%"后面的所有内容是注释而非代码,不会输出到最后的PDF中
% 使用本模板,只需要参照输出的PDF,在本文档的相应位置做简单替换即可
% 修改之后,输出更新后的PDF,只需要点击Overleaf中的“Recompile”按钮即可

% 填写你的名字
\CN{
  \name{鲍有能}
}
\EN{
  \name{Youneng Bao}
}
%**********************************相关信息****************************************
% \otherInfo后面的四个大括号里的所有信息都会在一行输出,最多使用四个大括号,填写四个信息
% 如果选择不填信息,那么大括号必须空着不写,而不能删除大括号。
% 如果想要把信息写两行,那就用两次指令\otherInfo{}{}{}{}即可
\CN{
  \info{手机/微信:18098861727}{邮箱:baoyouneng@163.com}{}{}
  \info{性别:男}{出生年月:1994年4月}{}{}{}
  % \info{政治面貌:中共党员}{籍贯:湖北省黄冈市}{}{}
  % \info{学术主页:https://scholar.google.com/citations?hl=en&user=lKD67zcAAAAJ}
  \info{研究方向:视觉数据压缩理论与系统,AI编码系统的安全和部署}{}{}{}
  % \info{专业及代码:信息与通信工程~0810}{}{}{}
  % \info{研究方向:人工智能的部署以及鲁棒性}{}{}{}
}
\EN{
  \info{mobile: (+86) 18098861727}{baoyouneng@163.com}{}{}
  \info{Gender: Male}{Birthplace: Macheng, Hubei Province}{}{}
  \info{Current address: Shenzhen, Guangdong Province}{}{}{}
}

%*********************************照片**********************************************
%照片需要放到images文件夹下,名字必须是you.jpg,如果不需要照片可以注释掉此行命令
%0.15的意思是,照片的大小是0.15倍,调整大小,避免遮挡文字
\yourphoto{0.12}
%**********************************正文**********************************************


%***大标题,下面有横线做分割
%***一般的标题有:教育背景,实习(项目)经历,工作经历,自我评价,求职意向,等等
\CN{
  \section{\faGraduationCap \ 教育背景}
}
\EN{
  \section{EDUCATION}
}

%***********一行子标题**************
%***第一个大括号里的内容向左对齐,第二个大括号里的内容向右对齐
%***\textbf{}括号里的字是粗体,\textit{}括号里的字是斜体
\CN{
  \datedsubsection{\textbf{哈尔滨工业大学(985工程),信息与通信工程,}\textbf{工学博士 \ \ 导师:\href{https://faculty.hitsz.edu.cn/liangyongsheng}{梁永生教授}}}{2020.09 - 2024.11}
}

\CN{
  \begin{itemize} [parsep=1ex]
    \item \textbf{毕业课题}:高效鲁棒的可学习图像压缩算法研究
  \end{itemize}
}


\CN{
  \datedsubsection{\textbf{大连理工大学(985工程),工学硕士 \ \ 导师:霍军周教授}}{2016.09 - 2019.07}
}



\CN{
  \datedsubsection{\textbf{哈尔滨工程大学(211工程),本科}}{2012.09 - 2016.07}
}


\CN{
  \section{\faBriefcase \ 工作经历}
}

\CN{
  \datedsubsection{\textbf{香港城市大学},博士后研究员,合作导师:\href{https://kedema.org/}{Kede Ma 教授}}{2025.01 - 至今}
  \CN{
  \begin{itemize} [parsep=1ex]
    \item \textbf{合作课题}:视觉大模型的Token压缩,高内涵(High-Content)三维细胞图像压缩
  \end{itemize}
}
}

\CN{
  \datedsubsection{\textbf{迈瑞医疗有限公司},智能系统工程师(全职)}{2019.09 - 2020.08}
}

\CN{
  \datedsubsection{\textbf{鹏城实验室},科研实习生}{2022.06 - 2024.06}
}


\CN{
  \section{\faBook  \ 论文/专利}
}

% \CN{
%   \datedsubsection{研究领域和兴趣}{}
% }


\CN{
  \datedsubsection{\textbf{科研论文(一作/共一论文: 10篇):~}{\href{https://scholar.google.com/citations?user=lKD67zcAAAAJ&hl=en}{\textbf{Google Scholar}}}
  ~{\href{https://dblp.org/pid/307/3082.html}{\textbf{DBLP}}}
  }{}

  % \vspace{0.3em}
  \begin{itemize}[label=\textcolor{NavyBlue}{\faFileText}, parsep=0.9ex]
    \item \textcolor{NavyBlue}{研究方向一:率失真优化理论与数据蒸馏}
  \end{itemize}
%  \vspace{0.3em}
  \begin{enumerate}[parsep=0.8ex]
    \item \textbf{Youneng Bao}, Yiping Liu, Zhuo Chen, Yongsheng Liang, Mu Li, Kede Ma$^{\dagger}$. Dataset Distillation as Data Compression: A Rate-Utility Perspective. In {Proceedings of the IEEE/CVF International Conference on Computer Vision, 2025.} (\textbf{CCF-A会议})
  \end{enumerate}
}


\begin{itemize}[label=\textcolor{NavyBlue}{\faRocket}, parsep=0.9ex]
  \item \textcolor{NavyBlue}{研究方向二:面向实用化部署的高效智能编码}
\end{itemize}


  \begin{enumerate}[parsep=1ex]
    \item \textbf{Youneng Bao}, Yulong Chen, Yiping Liu, Yichen Yang, Peng Qin, Mu Li, and Yongsheng Liang$^{\dagger}$. DynaQuant: Dynamic Mixed-Precision Quantization for Learned Image Compression. In {Proceedings of the AAAI Conference on Artificial Intelligence}, 2026. (\textbf{CCF-A会议})
    \item \textbf{Youneng Bao}, Fanyang Meng, Chao Li, Siwei Ma, Yonghong Tian, Yongsheng Liang$^{\dagger}$. Nonlinear Transforms in Learned Image Compression from a Communication Perspective. In {IEEE Transactions on Circuits and Systems for Video Technology}, 2022, 33(4): 1922-1936. (\textbf{JCR 1 区, 中科院 1 区top, IF = 8.4, CCF-B})
    \item \textbf{Youneng Bao}, Wen Tan, Chuanmin Jia, Mu Li, Yongsheng Liang$^{\dagger}$, Yonghong Tian. ShiftLIC: High-Efficiency Learned Image Compression with Spatial-Channel Shift Operations. In {IEEE Transactions on Circuits and Systems for Video Technology}, 2025. (\textbf{JCR 1 区, 中科院 1 区top, IF = 8.4, CCF-B})
    \item \textbf{Youneng Bao}, Wen Tan, Mu Li, Jiacong Chen, Qingyu Mao, Yongsheng Liang$^{\dagger}$. SFNIC: Hybrid Spatial-Frequency Information for Lightweight Neural Image Compression. In {CAAI Transactions on Intelligence Technology}, 2025. (\textbf{JCR 1 区, 中科院 1 区top, IF = 8.6})
    \item \textbf{Youneng Bao}, Wen Tan, Linfeng Zheng, Fanyang Meng, Wei Liu, Yongsheng Liang$^{\dagger}$. Taylor Series based Dual-Branch Transformation for Learned Image Compression. In {Signal Processing}, 2023, 212: 109128. (\textbf{JCR 2 区, 中科院 2 区, IF = 3.6, CCF-C})
    \item \textbf{Youneng Bao}, Chao Li, Fanyang Meng, Yongsheng Liang$^{\dagger}$, Wei Liu, Kaiyu Liu. MBB: A Multi-Scale Method for Data based on Bit Plane Slicing. In {Proceedings of {IEEE International Conference on Image Processing}}, 2021, 859-863. (\textbf{CCF-C会议})  
    \item  Mu Li, \textbf{Youneng Bao}, Xiaohang Sui, Jinxing Li$^{\dagger}$, Guangming Lu, Yong Xu. Learning Content-Weighted Pseudocylindrical Representation for 360° Image Compression. In {IEEE Transactions on Image Processing}, 2024, 33: 5975-5988. (\textbf{JCR 1 区, 中科院 1 区top, IF=10.8, CCF-A})

  \end{enumerate}

\vspace{0.3em}
\begin{itemize}[label=\textcolor{NavyBlue}{\faShield}, parsep=0.9ex]
  \item \textcolor{NavyBlue}{研究方向三:智能编码系统的鲁棒性与安全性}
\end{itemize}

  \begin{enumerate}[parsep=1ex]
    \item \textbf{Youneng Bao}, Wen Tan, Mu Li, Fanyang Meng, Yongsheng Liang$^{\dagger}$. Stable Successive Neural Image Compression via Coherent Demodulation-based Transformation[J]. Signal Processing, 2025, 227: 109741. (\textbf{JCR 2 区, 中科院 2 区, IF = 3.6, CCF-C}) 
    \item Zhi Cao, \textbf{Youneng Bao*}, Fanyang Meng, Chao Li, Wen Tan, Genhong Wang, and Yongsheng Liang$^{\dagger}$. Enhancing Adversarial Training with Prior Knowledge Distillation for Robust Image Compression[C]. In Proceedings of IEEE International Conference on Acoustics, Speech and Signal Processing, 2024: 3430--3434. (\textbf{共同一作, CCF-B})
    \item Can Luo, \textbf{Youneng Bao*}, Wen Tan, Chao Li, Fanyang Meng, and Yongsheng Liang$^{\dagger}$. A Complex-Valued Neural Network Based Robust Image Compression[C]. In Proceedings of Chinese Conference on Pattern Recognition and Computer Vision, 2023: 53--64. (\textbf{共同一作, CCF-C会议})  
  \end{enumerate}

% \CN{
%   \datedsubsection{\textbf{标准提案}}{}
%   \begin{enumerate}[parsep=0.5ex]
%     \item Algorithm Discription for Image Compression with Self-adaptive Computational Cost. IEEE1857.11 VC-35-M229, 
%   \end{enumerate}
% }

\CN{
  \datedsubsection{\textbf{发明专利,授权专利21项,其中学生一作4项}}{}
  \begin{itemize}[parsep=0.5ex]
    \item  梁永生,\textbf{鲍有能},谭文,李超。一种图像编码的方法、解码的方法及相关装置。国家发明专利,专利号:ZL202210496086.6。授权日期:2022.09.13
    \item 梁永生,\textbf{鲍有能},罗灿,谭文,李超。一种图像压缩方法、装置、设备及存储介质。国家发明专利,专利号:ZL20231.16678.8。授权日期:2023.05.30
  \end{itemize}
}


\CN{
\section{\faSearch\ 研究陈述}
}


\CN{
    \begin{itemize}[leftmargin=*, label={}]
        \item 围绕智能视觉数据压缩这一核心方向,我在博士及博士后阶段系统开展了以下研究工作:\par\par
        \item \vspace{0.3em}
        \noindent{\large \textcolor{NavyBlue}{\faFileText} \textbf{率失真优化理论与数据蒸馏}}
        \vspace{0.3em}        
        \item[] 从信息论视角重构数据蒸馏问题,建立``码率-数据集效用"的优化框架,为大规模数据集的高效处理提供了新思路:
        \begin{itemize}[leftmargin=1.5em, itemsep=0.2em]
            \item 创新点:(1)提出了基于率-效用权衡的数据集蒸馏新范式,将数据蒸馏问题转化为率失真优化问题,为数据集压缩提供了理论指导;(2)建立了首个端到端的``码率-数据集效用"蒸馏框架,实现了数据集大小与模型精度的最优权衡。
            \item 成果:相关工作发表在ICCV 2025。
        \end{itemize}

        \item \vspace{0.3em}
        \noindent{\large \textcolor{NavyBlue}{\faRocket} \textbf{面向实用化部署的高效智能编码}}
        \vspace{0.3em}
        
        \item[] 针对智能编码算法计算效率低的问题,开发了轻量化智能编码算法,将计算效率提升约5倍,为实用化部署奠定基础。
        \begin{itemize}[leftmargin=1.5em, itemsep=0.2em]
          \item 创新点:(1)引入泰勒级数分解的双分支变换,建立了非线性变换的数学基础;(2)提出了基于通信理论的非线性变换;(3)设计了空间-通道位移操作以及空间-频域双域特征融合方法,
          \item 成果:相关工作发表在IEEE TCSVT,Signal Processing等国际期刊。
        \end{itemize}
        
        \vspace{0.3em}
        \noindent{\large \textcolor{NavyBlue}{\faShield} \textbf{智能编码系统的鲁棒性与安全性}}
        \vspace{0.3em}
        
        \item[] 理论分析了智能编码系统的鲁棒性问题,构建了抗噪声干扰的鲁棒编码算法,为AI编码系统的可靠部署提供了理论保障。
        \begin{itemize}[leftmargin=1.5em, itemsep=0.2em]
            \item 创新点:(1)提出了基于相干解调的稳定编码框架;(2)提出了基于复值神经网络的神经压缩算法;
            \item 成果:相关工作发表在Signal Processing、ICASSP等国际期刊和会议。
        \end{itemize}

        \vspace{0.3em}
        \noindent{\large \textcolor{NavyBlue}{\faLightbulbO} \textbf{未来研究展望}}
        \vspace{0.3em}
        
        \item[] 未来,我将进一步拓展研究的广度与深度:将智能压缩框架延伸至医疗影像领域,实现TB级医学数据的高效压缩;融合生成式模型与语义通信理论,构建端到端的内容感知压缩体系;探索大语言模型与视觉编码的协同设计,构建统一的多模态压缩系统;基于信息论和机器学习理论,提升AI编码系统的可解释性与可信度。致力于为新一代可信、可解释且高效的智能编码系统提供核心理论支撑和技术突破。
    \end{itemize}
}




\CN{
\section{\faFlask \ 科研项目经历}
}

\CN{
  \datedsubsection{\textbf{国家自然科学基金面上项目},基于内容的智能视频编码优化研究}{2019.01-2022.12}
\begin{itemize}[parsep=0.8ex]
  \item 负责项目的中期报告、年度报告、结题报告的撰写;
\end{itemize}
}

\CN{
  \datedsubsection{\textbf{国家自然科学基金重点项目},超高清视频编码高效算法与芯片架构}{2021.01-2025.12}
  \begin{itemize}[parsep=0.8ex]
    \item 参与项目申请书的撰写,答辩PPT的撰写;
    \item 负责项目中神经网络图像编码部分的研究 (博士课题);
    \item 负责项目中期报告、年度报告、结题报告的材料撰写;
  \end{itemize}
}

\CN{
  \datedsubsection{\textbf{广东省重点建设学科项目},基于神经辐射场的高效视频编码技术研究}{2023.01-2025.12}
\begin{itemize}[parsep=0.8ex]
  \item 参与项目申请书,中期检查报告的撰写;
  \item 指导硕士生完成了神经辐射场高效表示、码率调节部分的研究,获得两项专利授权;
\end{itemize}
}


\CN{
\section{\faBullhorn \  指导与教学经历}
}

\CN{
  \begin{itemize}[parsep=0.8ex]
    \item \textbf{硕士生科研指导}:作为项目核心成员,累计指导/合作指导硕士研究生6名,围绕AI编码、神经辐射场等前沿方向开展研究。通过定期讨论与实验指导,所指导学生的相关成果已发表于CCF系列会议(如ICASSP, PRCV),并获得发明专利授权4项。
  \end{itemize}
}

\CN{
\section{\faTrophy \  荣誉奖项}
}

\CN{
  \begin{itemize}[parsep=0.8ex]
    \item 国家奖学金、校级/院级奖学金
  \end{itemize}
}

\CN{
\section{\faUsers \  学术服务}
}
\CN{
  \begin{itemize}[parsep=0.8ex]
    \item \textbf{期刊审稿人}:IEEE TIP, TCSVT, TMM, TCE
    \item \textbf{会议审稿人}:CVPR, ICCV, ICLR, AAAI, ACM MM, ICASSP, ICIP, PCS
  \end{itemize}
}




\CN{
\section{\faCogs \ 专业技能}
}
\CN{
    \begin{itemize}[parsep=0.8ex]
      \item \textbf{研究方向}:神经网络轻量化/部署,AI系统鲁棒性,生成模型,数据压缩
      \item \textbf{编程与框架}:Python, PyTorch, Matlab, C++
      \item \textbf{软件工具}:Git, Docker, LaTeX
    \end{itemize}
}



\end{document}
